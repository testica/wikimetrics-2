Un Wiki es un sitio web que puede ser modificado por cualquier persona, la finalidad de esto es que la misma comunidad aporte una mejora. Mayormente los wikis son de carácter informativo, un claro ejemplo es Wikipedia, que es la enciclopedia online más popular del mundo basado en wiki. Al ser un sitio que acepte colaboración se ve necesario a llevar un historial de los cambios realizados por los usuarios, logrando un control en las ediciones de estos. De una edición se pueden sacar algunas propiedades importantes como: el autor de la edición, que se representa con una dirección IP cuando es anónimo y con un nombre de usuario en caso contrario, la cantidad de texto editado y la fecha y hora en que realizó la edición. Hoy en día existen tantas personas colaborando en estos sitios que hacen que el historial de ediciones se haga muy extenso y difícil de comprender, la abundancia de datos provoca complejidad en su búsqueda e interpretación, lo que da lugar a la necesidad de un mecanismo que permite facilitar la comprensión y asimilación de la información, llamado visualización de datos.

La visualización de datos logra convertir un conjunto de datos complejo en una información clara y lo hace como su nombre indica, a través de elementos visuales, es decir, gráficas que juegan variedades de colores, figuras y texto. Es importante destacar que la visualización de datos necesita un estudio previo para la preparación y análisis de los datos.

Debido a lo expuesto anteriormente, en este trabajo se propone la implementación de un editor de visualizaciones de propiedades de historiales de wikis.

El presente trabajo se encuentra dividido en 3 capítulos, donde el primero habla sobre la visualización de datos y herramientas que explotan esta área para ayudarnos en el desarrollo. El segundo capítulo abarca las tecnologías web que contemplaremos para el desarrollo de la aplicación, como la arquitectura y diseño. Por último, en el cuarto capítulo se propone todo lo necesario para el desarrollo adecuado de la aplicación web.
